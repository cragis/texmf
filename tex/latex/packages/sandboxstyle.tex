\NeedsTeXFormat{LaTeX2e}[1994/06/01]
\ProvidesPackage{LabHW}[2018/03/15 CMH LabHW Package]

\RequirePackage{lmodern}
\RequirePackage[english]{babel}
\RequirePackage{fancyhdr}
\RequirePackage{lastpage}
\RequirePackage{tikz}
\RequirePackage{titling}
\RequirePackage{amsthm}
\RequirePackage{fp}
\RequirePackage{tabu}
\RequirePackage{adjustbox}
\RequirePackage[per-mode=symbol]{siunitx}
\RequirePackage[siunitx]{circuitikz}
\RequirePackage{indentfirst}
\RequirePackage{tabto}
\RequirePackage{graphicx}
\RequirePackage{csquotes}
\RequirePackage{needspace}
\RequirePackage{lastpage}

%% 'sans serif' option
\DeclareOption{sans}{
  \renewcommand{\familydefault}{\sfdefault}
}

%% 'roman' option
\DeclareOption{roman}{
  \renewcommand{\familydefault}{\rmdefault}
}

%% Global indentation option
\newif\if@neverindent\@neverindentfalse
\DeclareOption{neverindent}{
  \@neverindenttrue
}

\ExecuteOptions{roman}

\ProcessOptions\relax

%% Traditional LaTeX or TeX follows...
% ...

\newlength{\pardefault}
\setlength{\pardefault}{\parindent}
\newcommand{\neverindent}{ \setlength{\parindent}{0pt} }
\newcommand{\autoindent}{ \setlength{\parindent}{\pardefault} }

\if@neverindent
\neverindent
\fi

% ...




\geometry{top=.7in,
bottom=.5in,
headsep=0cm,
footskip=0.3cm,
  left=12mm, % left margin
  textwidth=150mm, % main text block
  marginparsep=2mm, % gutter between main text block and margin notes
  marginparwidth=50.mm % width of margin notes
}



\usetikzlibrary{decorations.text}
\usetikzlibrary{decorations.pathmorphing}

\usetikzlibrary{positioning}
\usetikzlibrary{patterns}
\usetikzlibrary{shapes}
\usetikzlibrary{plotmarks}






\MakeOuterQuote{"}

\newtheoremstyle{QS} %for questions leaving answer space after (the 2.5cm space below
{0pt} % Space above
{2 cm} % Space below
{} % Body font
{\parindent} % Indent amount
{\bfseries} % Theorem head font
{:} % Punctuation after theorem head
{0.5 em} % Space after theorem head
%\Needspace{3\baselineskip} 
{\thmname{#1}\thmnumber{#2}\thmnote{. #3}} % Theorem head spec (can be left empty, meaning `normal')





\newtheoremstyle{qS} %for questions leaving answer space after (the 0.5cm space below
{0pt} % Space above
{0.5cm} % Space below
{} % Body font
{\parindent} % Indent amount
{\bfseries} % Theorem head font
{:} % Punctuation after theorem head
{0.5 em} % Space after theorem head
{\thmname{#1}\thmnumber{#2}\thmnote{. #3}} % Theorem head spec (can be left empty, meaning `normal')  
  
  
\theoremstyle{QS}  
\newtheorem{Q}{HW}

\AtBeginEnvironment{Q}{\Needspace{3cm}}% \break if fewer than 3cm is available on page

\theoremstyle{qS}
\newtheorem{q}[Q]{HW}





%defining circuitikz bulb symbol with filament
\makeatletter
% used to process styles for to-path
\def\TikzBipolePath#1#2{\pgf@circ@bipole@path{#1}{#2}}
% restore size value for bipole definitions
\pgf@circ@Rlen = \pgfkeysvalueof{/tikz/circuitikz/bipoles/length}
\makeatother
\newlength{\ResUp} \newlength{\ResDown}
\newlength{\ResLeft} \newlength{\ResRight}

% bulb (lamp)
\ctikzset{bipoles/bulb/height/.initial=0.60}
\ctikzset{bipoles/bulb/width/.initial=.60}
\pgfcircdeclarebipole{}
{\ctikzvalof{bipoles/bulb/height}}
{bulb}
{\ctikzvalof{bipoles/bulb/height}}
{\ctikzvalof{bipoles/bulb/width}}
{
\pgfsetlinewidth{\pgfkeysvalueof{/tikz/circuitikz/bipoles/thickness}\pgfstartlinewidth}
\pgfextracty{\ResUp}{\northeast}
\pgfextractx{\ResRight}{\northeast}
\pgfextractx{\ResLeft}{\southwest}
\pgfextracty{\ResDown}{\southwest}
\pgfpathellipse{\pgfpointorigin}{\pgfpoint{0}{\ResUp}}{\pgfpoint{\ResLeft}{0}}
\pgfmoveto{\pgfpoint{\ResLeft}{0}}
\pgfpathcurveto{\pgfpoint{0}{0}}{\pgfpoint{.7\ResLeft}{.7\ResDown}}{\pgfpoint{0}{.5\ResDown}}
\pgfpathcurveto{\pgfpoint{\ResRight}{\ResUp}}{\pgfpoint{\ResLeft}{\ResUp}}{\pgfpoint{0}{-.5\ResUp}}
\pgfpathcurveto{\pgfpoint{.7\ResRight}{.7\ResDown}}{\pgfpoint{0}{0}}{\pgfpoint{-\ResLeft}{0}}
\pgfusepath{draw}
}
\def\bulbpath#1{\TikzBipolePath{bulb}{#1}}
\tikzset{bulb/.style = {\circuitikzbasekey, /tikz/to path=\bulbpath, l=#1}}

% bulb (lamp)
\ctikzset{bipoles/bulb2/height/.initial=0.60}
\ctikzset{bipoles/bulb2/width/.initial=.60}
\pgfcircdeclarebipole{}
{\ctikzvalof{bipoles/bulb2/height}}
{bulb2}
{\ctikzvalof{bipoles/bulb2/height}}
{\ctikzvalof{bipoles/bulb2/width}}
{
\pgfsetlinewidth{\pgfkeysvalueof{/tikz/circuitikz/bipoles/thickness}\pgfstartlinewidth}
\pgfextracty{\ResUp}{\northeast}
\pgfextractx{\ResRight}{\northeast}
\pgfextractx{\ResLeft}{\southwest}
\pgfextracty{\ResDown}{\southwest}
\pgfpathellipse{\pgfpointorigin}{\pgfpoint{0}{\ResUp}}{\pgfpoint{\ResLeft}{0}}
\pgfmoveto{\pgfpoint{\ResLeft}{0}}
\pgfpathcurveto{\pgfpoint{0}{0}}{\pgfpoint{.7\ResLeft}{.7\ResDown}}{\pgfpoint{0}{.5\ResDown}}
\pgfpathcurveto{\pgfpoint{\ResRight}{\ResUp}}{\pgfpoint{\ResLeft}{\ResUp}}{\pgfpoint{0}{-.5\ResUp}}
\pgfpathcurveto{\pgfpoint{.7\ResRight}{.7\ResDown}}{\pgfpoint{0}{0}}{\pgfpoint{-\ResLeft}{0}}
\pgfusepath{draw}
}
\def\bulb2path#1{\TikzBipolePath{bulb2}{#1}}
\tikzset{bulb2/.style = {\circuitikzbasekey, /tikz/to path=\bulb2path, l=#1}}




\def\parsedate #1:20#2#3#4#5#6#7#8\empty{#4#5/20#2#3}
\def\moddate#1{\expandafter\parsedate\pdffilemoddate{#1}\empty}\pagestyle{fancy} %page numbering and header
\fancyhf{}

\rhead{\flushright\theactivity\hspace{1cm}\thepage /\pageref{LastPage}}
\lfoot{}
\renewcommand{\headrulewidth}{0pt}
\newcommand{\theactivity}{NN}
\renewcommand{\theactivity}{\thetitle}
\fancypagestyle{plain}{%
\fancyhf{} % clear all header and footer fields
\lhead{\flushright\textbf{\large{\thetitle}}\tabto{13cm}{Name:\underline{\hspace{5cm}}}} %\rule{2in}{.5pt}
\fancyfoot[L]{\tiny{C. Howald, Marietta College (\moddate{\jobname.tex})}} % except the footer
%\fancyfoot[R]{\small{more on reverse $\rightarrow$}} % except the footer
\renewcommand{\headrulewidth}{0pt}
\renewcommand{\footrulewidth}{0pt}}


\def\bulb#1#2{
\begin{scope}[shift={#1}, rotate=#2,scale=.25]
    \begin{scope}[shift={(0,3.7)}]
    	\draw[fill=white] (0,1) circle (2cm);
		\draw[white,fill=white] (-1.1,-.6) rectangle (1.1,-1.7);
		\draw (-1.1,-.67) to (-1.1,-1.8);
		\draw (1.1,-.67) to (1.1,-1.8);
		\draw (-.3,-3.4) rectangle (.3,-3.7);
		\draw[fill=white] (0,-3) circle [x radius=1cm, y radius=0.5cm];
		\draw[white,fill=white] (-.3,-3.4) rectangle (.3,-3.7);
		\draw (-.3,-3.46) rectangle (.3,-3.7);
		\draw[fill=white] (-1.1,-1.7) rectangle (1.1,-3);
		%\draw (-1,-7) rectangle (1,-4);
		%\draw (-.2,-4) rectangle (.2,-3.8);
		%\draw (0,-3.7) to (0,-3.8);
	\end{scope}
\end{scope}
}

\def\battery#1#2{
\begin{scope}[shift={#1}, rotate=#2]
    \begin{scope}[shift={(0,0)}]

		\draw (-.33,0) rectangle (.33,.933);
		\draw (-.066,.933) rectangle (.066,1);
		%\draw (0,-3.7) to (0,-3.8);
	\end{scope}
\end{scope}
}

\def\switch#1#2{
\begin{scope}[shift={#1}, rotate=#2]
    \begin{scope}[shift={(0,0)}]

		
		\draw (.8,-.15) rectangle (1,.15);
		\begin{scope}[rotate=30]
			\draw (0,-.1) rectangle (1.3,.1);
		\end{scope}
		\draw[fill=white] (0,-.15) rectangle (.2,.15);
		\draw (-.1,-.35) rectangle (1.1,-.15);
	\end{scope}
\end{scope}
}

\endinput