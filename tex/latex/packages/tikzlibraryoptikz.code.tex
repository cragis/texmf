\def\optsetxandyvecbyvalueofkey#1{
  \optsetxvecbyvalueofkey{#1}
  \optsetyvecbyvalueofkey{#1}
}
\def\optsetxvecbyvalueofkey#1{
  \pgfsetxvec{\pgfpoint{\pgfkeysvalueof{#1}}{0pt}}
}
\def\optsetyvecbyvalueofkey#1{
  \pgfsetyvec{\pgfpoint{0pt}{\pgfkeysvalueof{#1}}}
}

% Todo: find out whether this is the right way to introduce keys
% Todo: merge generic height / width with lenses
\pgfkeyssetvalue{/tikz/waist}{0.5cm} % Standard waist
\pgfkeyssetvalue{/tikz/optics diameter}{1cm} % Height of lenses / plates etc
\pgfkeyssetvalue{/tikz/beam splitter width}{0.75 cm} % Height of cubes
\pgfkeyssetvalue{/tikz/flip beam splitter axis}{0} % 1 to flip diagonal 
\pgfkeyssetvalue{/tikz/optical element width}{3cm} % Generic width
\pgfkeyssetvalue{/tikz/optical element height}{1cm} % Generic height


% Define shapes


% Vertical optical elements are vertical on paper (i.e. a vertical
% waveplate), that is the beam passes horizontally through it

% Vertical optical element
\pgfdeclareshape{vertical optical element}{
  \savedanchor{\southwest}{%
    \optsetxvecbyvalueofkey{/tikz/optical element width}
    \optsetyvecbyvalueofkey{/tikz/optical element height}
    \pgfpointxy{-0.5}{-0.5}}
  \savedanchor{\northeast}{%
    \optsetxvecbyvalueofkey{/tikz/optical element width}
    \optsetyvecbyvalueofkey{/tikz/optical element height}
    \pgfpointxy{0.5}{0.5}}
  \savedanchor{\beamnorth}{
    \pgfpointorigin
    \pgf@xa = \pgf@x \pgf@ya=\pgf@y \advance\pgf@ya by
    \pgfkeysvalueof{/tikz/waist}
    \pgfpoint{\pgf@xa}{\pgf@ya}
  }
  
  \anchor{north}{
    % store lower left in xa/ya and upper right in xb/yb 
    \southwest \pgf@xa=\pgf@x \pgf@ya=\pgf@y 
    \northeast \pgf@xb=\pgf@x \pgf@yb=\pgf@y
    \pgfpoint{0}{\pgf@yb}
  }
  \anchor{north west}{
    % store lower left in xa/ya and upper right in xb/yb 
    \southwest \pgf@xa=\pgf@x \pgf@ya=\pgf@y 
    \northeast \pgf@xb=\pgf@x \pgf@yb=\pgf@y
    \pgfpoint{\pgf@xa}{\pgf@yb}
  }
  \anchor{west}{
    % store lower left in xa/ya and upper right in xb/yb 
    \southwest \pgf@xa=\pgf@x \pgf@ya=\pgf@y 
    \northeast \pgf@xb=\pgf@x \pgf@yb=\pgf@y
    \pgfpoint{\pgf@xa}{0}
  }
  \anchor{south west}{\southwest}
  \anchor{south}{
    % store lower left in xa/ya and upper right in xb/yb 
    \southwest \pgf@xa=\pgf@x \pgf@ya=\pgf@y 
    \northeast \pgf@xb=\pgf@x \pgf@yb=\pgf@y
    \pgfpoint{0}{\pgf@ya}
  }
  \anchor{south east}{
    % store lower left in xa/ya and upper right in xb/yb 
    \southwest \pgf@xa=\pgf@x \pgf@ya=\pgf@y 
    \northeast \pgf@xb=\pgf@x \pgf@yb=\pgf@y
    \pgfpoint{\pgf@xb}{\pgf@ya}
  }
  \anchor{east}{
    % store lower left in xa/ya and upper right in xb/yb 
    \southwest \pgf@xa=\pgf@x \pgf@ya=\pgf@y 
    \northeast \pgf@xb=\pgf@x \pgf@yb=\pgf@y
    \pgfpoint{\pgf@xb}{0}
  }
  \anchor{north east}{\northeast}
  \anchor{center}{\pgfpointorigin}

  \anchor{beam north}{\beamnorth}
  
  \anchor{beam north west}{%
    \southwest \pgf@xa=\pgf@x
    \beamnorth \pgf@yc = \pgf@y
    \pgfpointorigin
    \pgf@ya = \pgf@y \advance\pgf@ya by \pgf@yc
    \pgfpoint{\pgf@xa}{\pgf@ya}
  }
  \anchor{beam south west}{%
    \southwest \pgf@xa=\pgf@x
    \beamnorth \pgf@yc = \pgf@y
    \pgfpointorigin
    \pgf@ya = \pgf@y \advance\pgf@ya by -\pgf@yc
    \pgfpoint{\pgf@xa}{\pgf@ya}
  }
  \anchor{beam south}{
    \beamnorth \pgf@ya=-\pgf@y
    \pgfpointorigin \pgf@xa = \pgf@x
    \pgfpoint{\pgf@xa}{\pgf@ya}
  }
  \anchor{beam south east}{%
    \northeast \pgf@xa=\pgf@x
    \beamnorth \pgf@yc = \pgf@y
    \pgfpointorigin
    \pgf@ya = \pgf@y \advance\pgf@ya by -\pgf@yc
    \pgfpoint{\pgf@xa}{\pgf@ya}
  }
  \anchor{beam north east}{
    \northeast \pgf@xa=\pgf@x
    \beamnorth \pgf@yc = \pgf@y
    \pgfpointorigin
    \pgf@ya = \pgf@y \advance\pgf@ya by \pgf@yc
    \pgfpoint{\pgf@xa}{\pgf@ya}
  }
  
  \backgroundpath{
    % No rounded corners 
    \pgfsetcornersarced{\pgfpointorigin}
    \pgfpathrectanglecorners{\northeast}{\southwest}
  }
}

% Vertical wave plate
\tikzset{vertical wave plate/.style={vertical optical element,
    optical element height=\pgfkeysvalueof{/tikz/optics diameter},
    optical element width=0.1*\pgfkeysvalueof{/tikz/optics diameter}}}

\pgfdeclareshape{vertical cavity}{
  \savedanchor{\southwest}{%
    \optsetxvecbyvalueofkey{/tikz/optical element width}
    \optsetyvecbyvalueofkey{/tikz/optical element height}
    \pgfpointxy{-0.5}{-0.5}}
  \savedanchor{\innersouthwest}{%
    \optsetxvecbyvalueofkey{/tikz/optical element width}
    \optsetyvecbyvalueofkey{/tikz/optical element height}
    \pgfpointxy{-0.4}{-0.5}}
  \savedanchor{\northeast}{%
    \optsetxvecbyvalueofkey{/tikz/optical element width}
    \optsetyvecbyvalueofkey{/tikz/optical element height}
    \pgfpointxy{0.5}{0.5}}
  \savedanchor{\innernortheast}{%
    \optsetxvecbyvalueofkey{/tikz/optical element width}
    \optsetyvecbyvalueofkey{/tikz/optical element height}
    \pgfpointxy{0.4}{0.5}}
  \savedanchor{\beamnorth}{
    \pgfpointorigin
    \pgf@xa = \pgf@x \pgf@ya=\pgf@y \advance\pgf@ya by
    \pgfkeysvalueof{/tikz/waist}
    \pgfpoint{\pgf@xa}{\pgf@ya}
  }
  \inheritanchor[from=vertical optical element]{north}
  \inheritanchor[from=vertical optical element]{north west}
  \inheritanchor[from=vertical optical element]{west}
  \inheritanchor[from=vertical optical element]{south west}
  \inheritanchor[from=vertical optical element]{south}
  \inheritanchor[from=vertical optical element]{south east}
  \inheritanchor[from=vertical optical element]{east}
  \inheritanchor[from=vertical optical element]{north east}
  \inheritanchor[from=vertical optical element]{center}
  \inheritanchor[from=vertical optical element]{beam north}
  \inheritanchor[from=vertical optical element]{beam north west}
  \inheritanchor[from=vertical optical element]{beam south west}
  \inheritanchor[from=vertical optical element]{beam south}
  \inheritanchor[from=vertical optical element]{beam south east}
  \inheritanchor[from=vertical optical element]{beam north east}

  \backgroundpath{
    % No rounded corners
    \pgfsetcornersarced{\pgfpointorigin}
    \optsetxvecbyvalueofkey{/tikz/optical element width}
    \optsetyvecbyvalueofkey{/tikz/optical element height}

    % Calculate mirror params
    % Todo make them keys
    \newdimen\curvature
    \pgfextractx{\curvature}{\pgfpointxy{0.75}{0}}
    \newdimen\mirrorheight
    \pgfextracty{\mirrorheight}{\pgfpointxy{0}{1}}
    \def\cavityangle{asin(\mirrorheight/(2*\curvature))}
    
    \pgfpathmoveto{\southwest}
    \pgfpathlineto{\innersouthwest}
    \pgfpatharc{180+\cavityangle}{180-\cavityangle}{\curvature}
    \northeast \pgf@ya = \pgf@y
    \southwest \pgf@xa = \pgf@x
    \pgfpathlineto{\pgfpoint{\pgf@xa}{\pgf@ya}}
    \pgfpathclose

    \pgfpathmoveto{\northeast}
    \pgfpathlineto{\innernortheast}
    \pgfpatharc{\cavityangle}{-\cavityangle}{\curvature}
    \northeast \pgf@xa = \pgf@x
    \southwest \pgf@ya = \pgf@y
    \pgfpathlineto{\pgfpoint{\pgf@xa}{\pgf@ya}}
    \pgfpathclose

    
  }

}

% Lens
% Todo: Find out how width of lens scales nicely if height is changed
% Todo: make it inherit the nodes from vertical optical element where possible
\pgfdeclareshape{vertical lens}{%
  \savedanchor{\centerpoint}{\pgfpointorigin}
  \savedanchor{\semiminor}{%
    \optsetxandyvecbyvalueofkey{/tikz/optics diameter} \pgfpointxy{0.2}{0}}
  \savedanchor{\semimajor}{%
    \optsetxandyvecbyvalueofkey{/tikz/optics diameter} \pgfpointxy{0}{0.5}}

  \savedanchor{\beamnorth}{%
    \pgfpointorigin
    \pgf@xa = \pgf@x
    \pgf@ya = \pgf@y \advance\pgf@ya by \pgfkeysvalueof{/tikz/waist}
    \pgfpoint{\pgf@xa}{\pgf@ya}}

  \anchor{center}{\centerpoint}    
  \anchor{north}{\semimajor}
  \anchor{east}{\semiminor}
  \anchor{west}{%
    \pgf@process{\semiminor} \pgf@xa = -\pgf@x \pgf@ya=\pgf@y
    \pgfpoint{\pgf@xa}{\pgf@ya}
  }
  \anchor{south}{%
    \pgf@process{\semimajor} \pgf@xa = \pgf@x \pgf@ya =-\pgf@y
    \pgfpoint{\pgf@xa}{\pgf@ya}
  }

  \anchor{beam north}{\beamnorth}
  \anchor{beam south}{
    \beamnorth \pgf@xa=\pgf@x \pgf@ya = -\pgf@y
    \pgfpoint{\pgf@xa}{\pgf@ya}
  }

  \backgroundpath{
    % No rounded corners 
    \pgfsetcornersarced{\pgfpointorigin}
    \semimajor \pgf@xa =\pgf@x \pgf@ya = \pgf@y
    \semiminor \pgf@xb = \pgf@x \pgf@yb = \pgf@y

    % Calculate x and y coordinates of support for bezier curve
    \pgf@xc = \pgf@xb \advance \pgf@xc by 0.1\pgf@xb
    \pgf@yc = 0.5\pgf@ya
    
    % Draw first half of the lens
    \pgfpathmoveto{\pgfpoint{0}{-\pgf@ya}}
    \pgfpathcurveto{\pgfpoint{\pgf@xc}{-\pgf@yc}}
    {\pgfpoint{\pgf@xc}{\pgf@yc}}{\pgfpoint{\pgf@xa}{\pgf@ya}}

    % Todo: get rid of this repition, somehow ya gets overwritten by the
    % above path
    \semimajor \pgf@xa =\pgf@x \pgf@ya = \pgf@y
    \semiminor \pgf@xb = \pgf@x \pgf@yb = \pgf@y

    % Calculate x and y coordinates of support for bezier curve
    \pgf@xc = \pgf@xb \advance \pgf@xc by 0.1\pgf@xb
    \pgf@yc = 0.5\pgf@ya

    % Draw second half of lens
    \pgfpathcurveto{\pgfpoint{-\pgf@xc}{\pgf@yc}}
    {\pgfpoint{-\pgf@xc}{-\pgf@yc}}{\pgfpoint{0}{-\pgf@ya}}
    \pgfpathclose
  }
}


% Horizontal optical elements are horizontal on paper (i.e. a horizontal
% waveplate), that is the beam passes vertically through it


% Cross optical elements are ones through which the beam may pass both
% horizontally and vertically

% Beamsplitter
% Todo respect minimum width
\pgfdeclareshape{beam splitter}{%
  \savedanchor{\southwest}{%
    \optsetxandyvecbyvalueofkey{/tikz/beam splitter width} \pgfpointxy{-0.5}{-0.5}}
  \savedanchor{\northeast}{%
    \optsetxandyvecbyvalueofkey{/tikz/beam splitter width} \pgfpointxy{0.5}{0.5}}
  \savedanchor{\beamnee}{%
    \optsetxandyvecbyvalueofkey{/tikz/beam splitter width} \pgfpointxy{0.5}{0.5}
    \pgf@xa = \pgf@x 
    \pgfpointorigin
    \pgf@ya = \pgf@y \advance\pgf@ya by \pgfkeysvalueof{/tikz/waist}
    \pgfpoint{\pgf@xa}{\pgf@ya}}
  
  \inheritanchor[from=vertical optical element]{north}
  \inheritanchor[from=vertical optical element]{north west}
  \inheritanchor[from=vertical optical element]{west}
  \inheritanchor[from=vertical optical element]{south west}
  \inheritanchor[from=vertical optical element]{south}
  \inheritanchor[from=vertical optical element]{south east}
  \inheritanchor[from=vertical optical element]{east}
  \inheritanchor[from=vertical optical element]{north east}
  \inheritanchor[from=vertical optical element]{center}
  
  % Define anchors where the beam hits variaus parts of the cube
  \anchor{beam nee}{%
    \beamnee
  }
  
  \anchor{beam nne}{%
    \northeast \pgf@ya=\pgf@y
    \beamnee \pgf@yc = \pgf@y
    \pgfpointorigin
    \pgf@xa = \pgf@x \advance\pgf@xa by \pgf@yc
    \pgfpoint{\pgf@xa}{\pgf@ya}}
  
  \anchor{beam nnw}{%
    \southwest \pgf@xa=\pgf@x
    \beamnee \pgf@yc = \pgf@y
    \pgfpointorigin
    \pgf@ya = \pgf@y \advance\pgf@ya by \pgf@yc
    \pgfpoint{\pgf@xa}{\pgf@ya}}

  \anchor{beam nww}{%
    \southwest \pgf@xa=\pgf@x
    \beamnee \pgf@yc = \pgf@y
    \pgfpointorigin
    \pgf@ya = \pgf@y \advance\pgf@ya by \pgf@yc
    \pgfpoint{\pgf@xa}{\pgf@ya}}

  \anchor{beam sww}{%
    \southwest \pgf@xa=\pgf@x
    \beamnee \pgf@yc = \pgf@y
    \pgfpointorigin
    \pgf@ya = \pgf@y \advance\pgf@ya by -\pgf@yc
    \pgfpoint{\pgf@xa}{\pgf@ya}}

  \anchor{beam ssw}{%
    \southwest \pgf@ya=\pgf@y
    \beamnee \pgf@yc = \pgf@y
    \pgfpointorigin
    \pgf@xa = \pgf@x \advance\pgf@xa by -\pgf@yc
    \pgfpoint{\pgf@xa}{\pgf@ya}}

  \anchor{beam sse}{%
    \southwest \pgf@ya=\pgf@y
    \beamnee \pgf@yc = \pgf@y
    \pgfpointorigin
    \pgf@xa = \pgf@x \advance\pgf@xa by \pgf@yc
    \pgfpoint{\pgf@xa}{\pgf@ya}}

  \anchor{beam see}{%
    \northeast \pgf@xa=\pgf@x
    \beamnee \pgf@yc = \pgf@y
    \pgfpointorigin
    \pgf@ya = \pgf@y \advance\pgf@ya by -\pgf@yc
    \pgfpoint{\pgf@xa}{\pgf@ya}}


  \backgroundpath{%
    % store lower left in xa/ya and upper right in xb/yb 
    \southwest \pgf@xa=\pgf@x \pgf@ya=\pgf@y 
    \northeast \pgf@xb=\pgf@x \pgf@yb=\pgf@y

    % No rounded corners (don't know how to connect
    % diagonal to rounded corners directly
    \pgfsetcornersarced{\pgfpointorigin}
    
    % Draw a diagonal line
    \ifnum \pgfkeysvalueof{/tikz/flip beam splitter axis} = 1
    \pgfpathmoveto{\pgfpoint{\pgf@xb}{\pgf@ya}}
    \pgfpathlineto{\pgfpoint{\pgf@xa}{\pgf@yb}}
    \else
    \pgfpathmoveto{\pgfpoint{\pgf@xa}{\pgf@ya}}
    \pgfpathlineto{\pgfpoint{\pgf@xb}{\pgf@yb}}
    \fi
    \pgfpathclose


    % Draw cube
    \pgfpathmoveto{\pgfpoint{\pgf@xb}{\pgf@yb}}
    \pgfpathlineto{\pgfpoint{\pgf@xb}{\pgf@ya}}
    \pgfpathlineto{\pgfpoint{\pgf@xa}{\pgf@ya}}
    \pgfpathlineto{\pgfpoint{\pgf@xa}{\pgf@yb}}
    \pgfpathclose
  }

}

%%% Local Variables: 
%%% mode: latex
%%% TeX-master: "optikztest"
%%% End: 
